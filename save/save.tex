%!TEXprogram=xelatex
\documentclass[UTF8]{ctexart}
\begin{document}

\end{document}

立体视觉兴起于20世纪60年代,在近几十年中一直是计算机视觉领域最热门的研究问题之一。立体视觉的基本原理是融合两个或者多个视点的信息,以获取不同视角下同一个物体的图像,通过三角测量原理计算图像像素间的位置偏差,从而获得物体的三维信息。立体视觉直接模拟了人类视觉处理景物的方式,可以获得场景的丰富信息,尤其是深度信息,通过这些年的研究,立体视觉理论不断发展,并且由于硬件设备性能不断提高,立体视觉发挥着越来越大的作用,目前已广泛应用于机器人导航、工业检测、视觉测量等领域。

立体视觉包括图像获取、摄像机定标、特征提取、立体匹配、深度确定及内插等几部分[1],其中立体匹配是立体视觉中最重要也是最难的一部分,立体匹配的目的就是寻求同一空间景物在不同视点下投影图像的像素间的一一对应关系,立体匹配算法必须解决以下3个问题:选择正确的匹配特征,寻找特征间的本质属性及建立能正确匹配所选特征的稳定算法。立体匹配在立体视觉中占有重要的作用,立体视觉的应用推动匹配算法不断的发展,尤其是在速度和精度两个方面,文章简要分析了匹配算法的分类,较详细地综述了匹配算法在实时性和准确性两个方面的进展。

在近几十年中,立体匹配算法有了很大的发展,到20世纪90年代,一些基本的匹配算法已经发展成熟,目前立体匹配的研究基本上分为两个方向:一方面从理解人类视觉的立体融合机制出发,试图建立一种通用的人类双眼视觉计算模型;
另一方面从实际应用和要求出发,建立实用的立体视觉系统。第一个方向是从视觉生理的角度揭示人类视觉的立体融合机制,从而建立通用的立体视觉系统,目前的立体视觉模型还有一定的局限性,需要对视觉模型进行进一步完善。而第二个方向是针对不同的应用和要求,以建立更直接的专用的和面向对象的立体视觉系统为目的,通过强调场景和任务的约束来降
低视觉处理问题的难度,提高实用性。
在机器人导航、微操作系统的参数检测、三维测量和虚拟现实等领域,立体视觉发挥着重大的作用,对其中的匹配算法要求比较高,而各种应用场合,对视觉系统的要求不同,如视觉测量和基于图像的着色等应用领域需要非常精确的三维信息,这方面的需求推动立体匹配算法向高精度方向发展,立体匹配算法需要考虑遮挡和图像纹理单一等特殊情况,并要获得稠密的深度值。
而对于机器人导航和机械臂的视觉伺服等领域,系统需要的更重要的指标是实时性,机器人的视觉系统并不需要精度很高的深度值,或者对物体边缘的一些深度值并不需要非常精确,但是这些场合通常需要立体视觉系统提供实时的三维信息,因此这些领域的匹配算法向实时性发展,在满足一定匹配精度的要求下,尽量减小算法的复杂度,提高匹配的速度。下面就这两个方面的应用介绍匹配算法在软件与硬件方面的发展。


在三维空间坐标系0。咒匕z。中,左摄像头D。和右摄像头D,处于理想位置关系。其中,左摄像头光轴互和右摄像头光轴z,平行,且直线互z,与三维空间坐标系中的丘轴垂直。2个成像中心O;、O,的连线为基线b。三维空间中的一点P与两成像中心点D,、D,以及映射点P。和P,处于同一平面,平面与两成像平面的交线被称为极线。根据立体视觉的几何成像原理,双目立体视觉系统中对第三维(即深度)的测量由式(1)得出。其中∥为焦距,d为视差,即点P,和p,两图像在x轴方向上的差值,深度为z。基线b和焦距/可在双目标定时计算得出,因此双目立体视觉定位的核心内容是对视差d的计算。视差d由立体匹配得出,故对立体匹配的研究和讨论成为双目视觉系

1)唯一性约束:三维空间中的点映射到左右摄像头时,只会在图像上映射出唯一的对应点。在匹配时左右图像只有唯一一点相匹配。
2)连续性约束:三维空间中的物体一般是连续光滑的,在映射到左右摄像头时这种特性也被保留下来。在连续的图像上,其视差也一般是连续的。
3)极线约束:如图1所示,对于一个图像上的映射点来说,其匹配点必定落在另一图像的极线上。而对双目视觉系统来说,通常将其校正为上文原理中的假设情况,在这种情况下匹配点对处于同一水平线,即点在图像中的坐标Y值相等。
4)顺序一致性约束:三维空间中点的位置关系会在映射到成像平面时保留下来,即原物体的位置顺序在两幅图像中不会改变。

匹配基元是立体匹配中的单位匹配对象,也就是立体匹配中的匹配特征对象。一般有以下几种,点:在双目立体匹配中最常用的特征就是点特征,例如像素点的灰度、颜色值、Harris角点、SIFT特征点、SURF特征点等;块:图像块也可以认为是图像区域,块的特征提取可以是直观的块内所有像素点的灰度或颜色值,也可以是对块内像素点的统计特征,还可以是对块内像素点的变换特征;线:线一般只是边缘线,是最能体现图像的纹理区域特征。在有些方法中将角点或特征点之间的连线作为特征线来提取。

传统标定方法需要先建立高精度标定物上某些已知的三维空间点和其对应的图像坐标间的映射关系,接着利用投影变化计算得出摄像机相关参数。B.Hallert最早将最小二乘法引入到摄像机标定中得出较高精度的标定结果,随后该方法被其成功应用到野外立体坐标测量仪上。Faig等综合考虑多种因素对相机成像模型的影响,在一幅图像上利用至少 17 个约束参数来描述摄像机成像模型,最后利用非线性优化对所有参数进行求解,非线性优化思想的引入一方面造就了高精度的标定结果,但是另一方面也带来高额的计算代价,同时算法过于依赖算法初始值的选取,导致该算法在实时应用领域推广甚少。Tsai在将传统的线性标定和非线性标定相结合后提出了经典的两步标定法,该算法首先利用直接线性变换的方式预估摄像机的模型参数,将线性求解出的模型参数作为初值,然后加入径向畸变因子(该算法忽略了切向畸变),最后通过非线性优化对之前的摄像机参数和径向畸变因子求解。该算法只对部分参数进行非线性优化求解带来的好处是计算速度明显加快,但是其简单的畸变模型使其在处理大视场和具有严重镜头畸变的情况时效果欠佳。J.Weng等人为了克服 Tsai 标定法的缺点,进一步完善了该方法的畸变模型,使得算法的应用范围进一步扩大,能够较好的处理严重畸变。微软亚洲研究院的张正友研究员提出一种利用平面模板标定摄像机的简单方法,该方法不需要高精度的三维标定物,而是利用一个棋盘格图案的平面标定板就可完成对摄像机的标定工作,极大的简化了摄像机标定物的制作,因此该方法得到了广泛的应用。

只要能够找到空间中某点在左右两个摄像机像面上的相应点,并且通过摄像机标 定获得摄像机的内外参数,就可以确定这个点的三维坐标

计算机视觉是使用计算机及相关设备对生物视觉的一种模拟。它的主要任务是通过对采集的图片或视频进行处理以获得相应场景的三维信息。

立体视觉匹配是计算机视觉中一个非常重要又很困难的问题,它的目标是从不同视点图像中找到匹配的对应点。

人类的双目立体视觉系统是一个非常智能的系统。场景中的光线在人眼这个精密的成像系统中被采集,通过神经中枢被送入包含有数以亿计的神经元的大脑中被并行处理

在这四个主要模块中,图像获取方式主要取决于具体运用环境和系统要实现的目标,随着图像采集设备的日趋先进,图像获取的质量也逐步提高,图像获取环节早已不再是视觉研究的瓶颈,因此该部分的研究相对较少。深度计算部分主要根据前期求得的视差结合三角法进行三维信息的获取,该部分理论较为成熟

视觉是人类观察和认知外部世界的重要手段,人类获得的外部信息中有80%是通过视觉器官获取的[1][2]。经过一个相当漫长的过程,人类对视觉的认识从感性认识转到定性定量的分析研究。用机器代替人眼进行目标对象的识别、判断和测量是人类很多年的梦想,机器和计算机具有类似于人类的视觉感受能力,就能给人类提供更优质、更人性化、更接近人类水平的服务。

立体视觉系统的基本原理是融合两个或者多个视点的信息,以获取不同视角下同一个物体的图像,通过视差原理计算图像像素间的位置偏差,从而获得物体的三维信息。

摄像机标定是将数字图像上的像素点与世界坐标下的三维空间点进行对应,通过一定的算法求出摄像机的焦距、主点坐标、透镜畸变等内部参数,在获取内部参数之后经过一定变换可求出摄像机相对世界坐标系的位移和旋转等外部参数。

编写程序读取双目摄像头图像,完成摄像头离线标定,并根据标定结果对图像矫正,输出视差图。在该系统中要实现的功能有以下几点:
1、对摄像头进行离线标定,并根据标定结果矫正图像
2、双目立体匹配的基本功能要求
3、对比不同的立体匹配方法,分析实验结果

图像获取方式主要取决于具体运用环境和系统要实现的目标,随着图像采集设备的日趋先进,图像获取的质量也逐步提高,图像获取环节早已不再是视觉研究的瓶颈。
深度计算部分主要根据前期求得的视差结合三角法进行三维信息的获取,该部分理论较为成熟。
因此立体匹配和摄像机标定是重点研究对象。

