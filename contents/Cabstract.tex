% !Mode:: "TeX:UTF-8"
% 关键词只取前五个值
\begin{Cabstract}{立体视觉}{摄像机标定}{局部匹配}{种子点传播}{平面约束}

双目立体视觉是计算机视觉的一个重要研究分支,它通过计算同一场景下空间点在两幅图像上的视差来恢复场景的三维深度信息。与传统的深度测量方式相比,它具有非接触性和被动性两个最大的优点。这些优点使其在工业检测、航空航天、机器人导航及战场监视等领域得到了广泛的运用。 

近年来,双目立体视觉方法虽然取得较大进展,但其中有些问题仍有待更好的解决,比如摄像机的标定精度和立体匹配的准确性问题,还需要研究人员不断提出新的解决办法。本文针对双目立体视觉中的摄像机标定、立体匹配、立体测量误差和精度分析等方面重点展开了理论与实现技术的研究。主要工作和创新点如下: 

1. 为了提高摄像机的标定精度,提出一种对初始角点有限邻域进行直线拟合
的角点提取算法。算法以初步提取的 Harris 角点为中心,在其有限邻域范围内
对棋盘格边缘进行直线拟合,将直线交点作为最终角点。利用提取的角点在平面
标定法下对立体视觉系统进行标定,获得了较高的标定精度,证明了该角点提取
方法的有效性。 

2. 为了较好的平衡局部立体匹配算法的准确度和速度,提出了一种基于种子
点传播的快速立体匹配算法。算法采用联合匹配代价度量左右图像差异,利用图
像边缘信息辅助构建动态匹配窗口,以削弱固定窗的不利影响来获取高质量的初
始视差图,然后通过左右一致检验和区域特性筛选出种子点,在多种约束下进行
种子点传播,最后采用区域投票优化视差图。算法在测试图对的大部分区域获得
了较高的匹配精度,具有一定的实用价值。 

3. 针对半全局算法中存在的路径规划不完全问题,提出一种基于视差平面
约束的半全局立体匹配算法。算法采用 FAST 特征点和 ORB 描述子相结合的方
式对左右视图匹配以获取稀疏视差图,利用图像分割对稀疏视差点聚类并将其传
播给同一分割区域内的其它点;然后对分割区域内的可信视差点进行平面拟合,
将平面拟合视差和初始视差的差值作为像素点的匹配代价约束,最后利用半全局
方法求取能量方程最优解。实验表明算法在保持高效性的同时能够较好的处理弱
纹理和结构重复区域,对于遮挡和深度不连续区域也能够获得较为稳定的匹配结
果。 

4. 从理论上详细分析了双目立体视觉系统中影响测量精度的一些因素,主
要对摄像机像元量化误差、结构参数误差与距离分辨率精度等三个方面展开了理
论推导,并给出了相应的结论。最后在不同的场景下测试了本文的两个算法,通
过理论推导和实验数据分析相结合的方法,以期对双目立体视觉测距平台的构建
提出一些可供参考的建议。
\end{Cabstract}
