% !Mode:: "TeX:UTF-8"
% 关键词只取前五个值
\begin{Cabstract}{立体视觉}{局部匹配方法}{算法优化}{}{}
立体匹配是计算机视觉的一个重要分支,作为立体视觉中研究最活跃的主题之一,
在三维场景重建、移动机器、对象识别、智能控制、三维测量等领域有广泛的应用场景。 

立体匹配的实质是以同一场景从不同角度拍摄的两幅图像作为输入,提取图
像对应的视差图。通过视差与深度的关系,进一步获取场景中物体的深度信息,
从而应用于各个领域。立体视觉技术的实现一般包括:图像获取,图像校准,立
体匹配以及三维重建四个步骤。立体匹配是最重要也是最困
难的一步。

本文在经典局部匹配算法 SAD 的基础上,分析算法的不足,进行算法优化,
提出了基于遮挡检测的SAD算法,并为检测出来的遮挡区域的像素应用本文提出的遮挡区域填充算
法,另外通过添加 Census 灰度变换和基于几何距离权重的算法,提高立体匹配的匹配效果。用标准测试图像作为对比,
将实验结果通过评估函数进行测评。测评结果证
明,优化算法提高了图像遮挡区域的匹配精度,同时整体提高了视差图的匹配质量。
\end{Cabstract}

