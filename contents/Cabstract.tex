% !Mode:: "TeX:UTF-8"
% 关键词只取前五个值
\begin{Cabstract}{立体视觉}{局部匹配方法}{算法优化}{}{}
作为计算机视觉的一个重要分支,立体匹配是立体视觉中研究最活跃的主题
之一,在三维场景重建、移动机器人、对象识别、智能控制、三维测量等领域得
到了广泛应用。 

立体匹配的实质是以同一场景从不同角度拍摄的两幅图像作为输入,提取图
像对应的视差图。通过视差与深度的关系,进一步获取场景中物体的深度信息,
从而应用于各个领域。立体视觉技术的实现一般包括:图像获取,图像校准,立
体匹配以及三维重建四个步骤。在以上四个步骤中,立体匹配是最重要也是最困
难的一步。一般地,立体匹配算法分为两大类:局部立体匹配算法和全局立体匹
配算法。通过对比分析这两种算法的优缺点,本文将重点研究和讨论局部立体匹
配算法。

本文在经典局部算法SAD的基础上,通过分析 ASW 算法的不足,进行了新
的优化和改进,提出了基于 LRC 遮挡检测的改进的 ASW 算法。通过对输入图像
进行自左向右和自右向左的双向匹配得到的左图和右图对应的两幅视差图进行
LRC 遮挡检测,并为检测出来的遮挡区域的像素应用本文提出的遮挡区域填充算
法,实现了立体匹配。 

为了验证算法的有效性,以 Middlebury 网站提供的标准测试图像作为输入,
将实验结果提交至 Middlebury 网站的测评系统进行测评。实验结果和测评结果证
明,优化算法提高了图像遮挡区域的匹配精度,同时整体提高了视差图的匹配质量。
\end{Cabstract}

