% !Mode:: "TeX:UTF-8"

时光飞逝,岁月如梭,转眼间,四年的本科大学生活即将结束,本次论文作为我在大学期间做的最后一次学习研究任务,经历多个月时间的查找资料,学习基础知识,组织实验,撰写论文,最终顺利完成。

值此我首先要感谢我的导师李宏亮教授和研究生孙文龙学长。李宏亮导师专业知识渊博,治学态度严
谨,工作一丝不苟、精益求精,具有诲人不倦的高尚师德,孙文龙学长具有和蔼可亲的人格魅力,为我耐心讲解各个知识点,给我解决问题提供了许多思路。

除此之外,能够顺利完成本次论文,也离不开大学四年母校和多位老师对我的指导,培养了我独立思考和自主学习的能力,在此一并感谢。
感谢535教研室各位学长的关心与帮助,使我能够顺利完成论文的实验部分,感谢248寝室同学在我论文书写期间对我的支持和帮助。

通过本次论文的学习,除了对本课题有了深刻的了解,对科研流程的初步熟悉之外,查阅大量的文献资料,提取相关内容,锻炼了我自学的能力,组织论文结构,进行课题探讨,设计实验提高了我的思维能力,总体而言,这次论文的完成过程使我各方面得到很大的提升,再次感谢所有给予我帮助的人!
