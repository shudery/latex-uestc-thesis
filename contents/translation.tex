% !Mode:: "TeX:UTF-8"

\chapter{基于优化的匹配算法高效获取视差图}

视觉匹配是一个寻找两幅输入图像之间相似性的问题,它是计算机视觉领域的一个关键难题,拥有广泛的实际应用场景,因此近几年对于视觉匹配的研究也越来越多。

立体匹配的实质是以同一场景从不同角度拍摄的两幅图像作为输入,提取图
像对应的视差图。通过视差与深度的关系,进一步获取场景中物体的深度信息,
从而应用于各个领域。立体视觉技术的实现一般包括:图像获取,图像校准,立
体匹配以及三维重建四个步骤。在以上四个步骤中,立体匹配是最重要也是最困
难的一步。一般地,立体匹配算法分为两大类:局部立体匹配算法和全局立体匹
配算法。通过对比分析这两种算法的优缺点,本文将重点研究和讨论局部立体匹
配算法。

本文在经典局部匹配算法 SAD 的基础上,通过分析算法的不足,进行了算法优化,
提出了基于遮挡检测的改进的SAD算法。通过对输入图像
进行自左向右和自右向左的双向匹配得到的左图和右图对应的两幅视差图进行
遮挡检测,并为检测出来的遮挡区域的像素应用本文提出的遮挡区域填充算
法,另外通过添加 Census 灰度变换和基于几何距离权重的算法,提高立体匹配的匹配效果。