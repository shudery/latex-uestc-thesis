% !Mode:: "TeX:UTF-8"

\chapter{Efficient determination of disparity map from
stereo images with modified Sum of Absolute
Differences (SAD) algorithm}

This paper proposes modification of the
conventional Sum of Absolute Differences (SAD) for performance
improvement in depth-map estimation from stereo images
captured by a camera in a stereo system. The conventional SAD
is commonly search in whole stereo images to find out the
difference in pixels between the left and right captured images,
and then obtains the corresponding disparity map and this may
lead to high elapsing time. In order to reduce the number of
searching pixels, the proposed modified SAD tries to estimate the
difference only from edge pixels which are referred as pixels-ofinterest
and bring significant information about depth map. The
number of pixels being searched is reduced to about 17\% on the
total pixels, hence the total elapsing time is saved up to around
89\% compared to that of the conventional SAD. This results is
promising for implementation of a real-time vision system.

Stereo matching is a problem to find correspondences
between two input images,. It is one of fundamental
computer vision problems with a wide range of applications,
and hence it has been extensively studied in the computer
vision field for many recent years . Stereo matching consists to
find for each point in the left image, its corresponding in the
right one. The difference between horizontal distances of these
points is the disparity. A disparity map consists of all the
possible disparity values in an image. Such a map is basically
a representation of the depth of the perceived scene. Therefore,
the disparity maps have been used to address efficiently
problems such as 3D reconstruction, positioning, mobile robot
navigation, obstacle avoidance and many other domains.

There are three broad classes of techniques, which have
been used for stereo matching: area-based, featurebased, and phase-based. SAD-based implementations
are the most favorable area-based techniques in real-time
stereo vision, since they can be straightforwardly implemented
in hardware. The calculations required in terms of design units
are simple, since only summations and absolute values are
performed. Parallel design units can be utilized in order to
handle various disparity ranges, in order to reduce the
computational time required. SAD correlation algorithm
can be applied to solve the problem automatically detect mold
applications in robot control and mapping. The research
results show that the SAD correlation algorithm can be a
potential replacement for SIFT method proposed by Lowe in the landmark selection problem. The purpose of the
SIFT method is quoted as a full-featured constant rate,
rotate, move but criticized the selection of stable
characteristics proved more effective when reproducing 3D or
service of navigation.

Typically, the disparity is computed as a shift to the left of
an image feature when it is viewed in the right image, it is
calculated by determining a measure of the SAD, that is used
to calculate disparities at each pixel in the right image.
After this SAD “match strength” has been computed for all
valid disparities, the disparity that yields the lowest SAD is
determined to be the disparity at that location in the right
image. Hence, the computation of a disparity map is
performed on all pixels of stereo images so this approach may
be influenced by an object's position and large elapsing time to
determine the corresponding points between the two images,
especially for huge size of captured images or frames.
In this paper, the proposed SAD algorithm tries to modify
the way of disparity determination by utilization of edge
information extracted from the captured stereo images. Since
edges operators used in the proposed method require fewer
computational load, the disparity map can be represented with
low elapsing time while maintaining reasonable quality of the
map reconfiguration.

The rest of this paper is organized as follows: Section 2
describes modifications in the proposed SAD method to
determine a disparity map. In Section 3, experimental results
obtained on real images are presented and discussed. Finally,
Section 4 concludes the paper with some remarks and future
works. 
