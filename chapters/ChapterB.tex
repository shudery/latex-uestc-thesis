% !Mode:: "TeX:UTF-8"

\chapter{双目视觉系统}

\section{双目视觉系统概述}
人类观察外部世界,能够感知物体的立体性,分辨所观察的场景的深度以及景物的远近,人眼获取的场景信息是立体信息。双目立体视觉系统是立体视觉最常见的系统,是一种模拟人类双眼视觉观察特性的计算机视觉系统。利用双目立体视觉系统,能够实现对待测场景目标的距离感知与被动测距,以及实现目标场景的三维信息恢复与重构。双目立体视觉系统是,通过借助摄像机获取的图像信息,计算世界坐标系下物体的位置和形状等信息。

单幅图像只能够提供图像中每单个点的视线,即垂直平面的世界坐标,不能得到在该视线上世界坐标系中
相应的三维空间点与探测垂直平面的距离,因此单幅图像所获取的信息只能提供世界坐标系
中某平面上二维测量,无法重构场景的三维几何信息,这是将3D的空间转换成2D的图像发生的信息损失。
双目立体视觉技术是如何来恢复这些三维信息,从而重构三维场景呢?通过采集同一时刻,同一位置两幅二维图像的图像信息,利用立体匹配技术,结合视差测距,摄像机标定等先验知识,获取物体的三维表面形状信息的技术,其工作原理主要是基于人类双眼视觉中视差测距的原理。

双目立体视觉实现三维重建的主要包括如下步骤:第一步,摄像机标定,获取成像模型所需内外部参数;第二步,用标定好的摄像机系统拍摄目标场景图像,采集二维信息;第三步,双目立体匹配,
寻找两幅二维图像中待测点的点对点对应关系,若为稠密
匹配则进入第四步,若为稀疏匹配则进入第五步实现稀疏点测距;第四步,生成视差图;
第五步,根据视差计算深度,实现场景的三维重建。按照功能可以将双目视觉系统分为四个主要的模块,如图\ref{2-3}所示。

\pic[h]{双目视觉系统的主要模块}{width=14cm}{2-3}

\section{双目视觉系统的基本原理}

人类获取目标深度信息的原理如图\ref{2-1}所示。图中,L和R分别代表人的左眼和右眼以及两眼的视网膜图像,B为两眼间的瞳距(一般人眼的瞳距约为6厘米),
P为观察的目标。由于瞳距的关系,两只眼睛同时观察同一目标,由于两只眼睛的观察
角度不同,所以会有左眼看到左边多一些、右眼看到右边多一些的情况。若两眼同时观
察目标点P,则点P分别投影成像于两眼,这两个投影是有位置差异的,这个差异就是
视差,场景中所有点的视差组成一幅图像,称为视差图。大脑对视差图进行处理,可以产生对
场景的立体感知。人眼视觉的这个立体感知的过程就是基于视差的原理,视差原理也是计算机视觉的原理基础。

\pic[h]{人类视觉感知深度的原理示意图}{width=11cm}{2-1}

基于人类视觉特性的研究是立体视觉技术的重要研究方向,趋近于人类视觉能力是立体视觉技术的研究目标目前,在计算机立体视觉技术的研究中,基于单目多焦距图像序列的立体视觉技术、基于人类感知特性的立体视觉
技术以及基于双目成像的立体视觉技术都有研究人员从事相应工作。根据环境条件需求
和重构效果需求的不同,选择的立体视觉技术也不同。其中,双目立体视觉技术是最接
近人类视觉系统、适应于机器实现且重构效果较好的立体视觉技术。

双目视觉系统模拟人的视觉,对物体按照一定的规格采集两幅图像,以此分析计算出视差图。如图\ref{2-2}所示,对物体的某一点的深度感知无法在只有单目摄像机时感知,当物体从P点移动到Q时,对于做摄像机而言物体没有发生变化,而右摄像机则记录下了深度的信息,根据几何成像的原理即刻得出物体深度变化信息。而左右摄像机匹配对应像素点的过程就是立体匹配方法,将在下一章详细介绍。

\pic[h]{双目摄像机获取像素点信息的示意图}{keepaspectratio=false,height=6cm}{2-2}

\section{摄像机的标定}

双目视觉系统根据摄像机获取的图像信息计算出待测点的深度信息,重建目标场景的表面三维形状与空间位置。三维空间点与其在两二维成像平面中的投影点是对应的,该对应关系取决于摄像机参数,这些参数需要通过实验和计算来确定,该过程就称为摄像机标定。实现三维重建首先需要采集准确的目标场景图像,摄像机标定正是获取准确图像的必要过程。

摄像机的标定最初在摄影测量学中研究和发展,摄影测量学提出了四个标定问题:
(1)绝对方位:确定两个坐标系统之间的转换或位置和方向的一系列传感器在绝对坐标系的坐标点的校正。
(2)相对方位:确定了相对位置和方向在两个摄像头从预测的校准点在现场。
(3)外部方位:确定位置和方向,相机在绝对坐标系统的投影在现场校准点。
(4)内部方位:确定内部几何相机,其中包括摄像机常数,位置的主要观点,并更正为透镜畸变。

这些标定问题源于由航拍照片创建地形地图的技术,是经典的摄影测量学问题。除了这四个基本标定问题,摄影测量学也涉及双目立体场景差异深度确定和立体影像重建。

摄像机标定包括摄像机成像系统内外几何及光学参数的标定和多个摄像机之间相对位置关系的标定,其目的是确定目标物体世界坐标与图像坐标之间的映射关系。摄像机标定的精度直接关系到计算机视觉系统的系统误差,是决定三维重建精度的关键因素,非常具有研究和改进意义。

图像上每点的坐标与空间中物体表面对应点的实际位置相关,这种位置上的映射关系是由摄像机的成像几何模型所唯一确定的。一般把这个几何模型的参数称作摄像机参数,计算这些参数的过程就是摄像机定标的过程。作为双目立体视觉技术赖以计算的基础,摄像机标定的精度直接影响整个系统的测量效果。



