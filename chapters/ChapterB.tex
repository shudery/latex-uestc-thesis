% !Mode:: "TeX:UTF-8"

\chapter{双目视觉系统}

\section{双目视觉系统概述}
双目视觉系统如同人的双眼一样,我们观察外部世界,感知到外界三维信息,对物体远近的位置信息进行判断,在大脑中加工这些信息以使我们掌握真实的客观世界形态。双目视觉系统对人类双眼的模拟就是依据于此,为了使计算机能够获得客观世界中的物体三维信息,就需要利用双目视觉系统,将图像的二维信息进行一系列处理,还原世界坐标下该物体的三维信息。

如果只有一幅图像,那么摄像机无法获取到图像中物体的深度信息,因为单一图像只能保存平面在某一时刻的信息,摄像机无法识别某一像素点在与摄像机平面的垂线方面上进行移动的过程,因此无法得到完整的物体信息,但是如果有两台摄像机,就可以捕捉到这些信息,双目视觉系统由此而来。

双目立体视觉获取深度信息前需要先得到图像对的视差图,视差图就是两幅输入图像之间像素点的偏移量信息图,根据这些偏移量信息和摄像机标定得到的内部参数就能求出深度信息。视差图的获取的基本过程一共包括三个步骤:摄像机标定,获取成像模型所需内外部参数;图像获取,用标定好的摄像机系统拍摄目标场景图像,采集二维信息;立体匹配方法,寻找输入图像对之间的对应像素点,输出视差图;最后就是根据视差计来算深度,获取物体深度信息。综上可以按照功能可以将双目视觉系统分为图像获取,摄像机标定,立体匹配,深度计算四个主要的模块\citeup{A-2}。如图\ref{2-3}所示。

\pic[h]{双目视觉系统的主要模块}{width=14cm}{2-3}

\section{双目视觉系统的基本原理}
\pic[h]{人类视觉感知深度的原理示意图}{width=11cm}{2-1}

如图\ref{2-1}所示为人通过双眼获取物体深度信息的原理图。图中,L表示人的左眼,R表示人的右眼,投影平面可以看做人的视网膜成像平面,B为两眼之间的瞳距,通常人双眼之间的距离大概是6厘米,
P为待观察目标。两眼在同一时刻观察同一目标时,由于两只眼睛的观察
角度不同,所以左右眼看到的图像不是完全相同的,两眼成像像素点之间的偏移量就是视差值。将观察到的图像所有像素点一一映射对应,可以将所有图像的视差值记录成像,这个就是视差图。大脑对通过对两眼观察到的图像信息进行处理,就可以对现实场景进行立体感知。双目视觉系统就是模拟人眼视觉立体感知过程的,视差原理是双目视觉系统的一个重要原理。根据人类视觉的特性,研究其成像原理,是双目视觉领域的重要研究方向,可以说双目视觉技术的实现目标就是实现接近于人的双目视觉系统和人的三维感知能力。

简单来说,双目视觉系统模拟人眼识别,首先获取图像信息,随后通过立体匹配方法输出图像对的视差图,利用视差信息和摄像机标定得到的内外部参数,经过一系列计算得出三维信息,双目视觉系统的数学模型将在下一小节展开介绍。

双目视觉系统模拟人的视觉,对物体按照一定的规格采集两幅图像,以此分析计算出视差图。如图\ref{2-2}所示为一个双目摄像机获取像素信息的简单示意图,PQ是一个像素点的移动轨迹,对物体的某一点的深度感知无法在只有单目摄像机时感知,当物体从P点移动到Q时,对于做摄像机而言物体没有发生变化,而右摄像机则记录下了深度的信息,根据几何成像的原理即可得出物体深度变化信息,如果只有单目摄像机就无法复原丢失的三维信息。左右摄像机匹配对应像素点的过程就是立体匹配方法,立体匹配方法作为本论文的重点研究内容将在下一章详细介绍。

\pic[h]{双目摄像机获取像素点信息的示意图}{keepaspectratio=false,height=7cm}{2-2}

\section{双目视觉的数学模型}

双目视觉系统是利用视差图和摄像机内外部参数求解图像深度信息的过程,本节将简单介绍双目视觉系统在提取所需信息,求解现实场景三维信息的过程。

\pic[h]{双目测距原理图}{keepaspectratio=false,height=6cm}{2-4}

为了方便理解和公式求导,将成像面翻转到目标点与视点之间,目标点与视点之间的连线在左右两个成像平面的交点就是该目标点在摄像机坐标系中所成的像,双目测距原理示意图如图\ref{2-4}所示。

在双目摄像机安装,调节完成之后,通过摄像机标定获取相关参数,然后通过双目摄像机同时拍摄在焦距范围内的待拍目标P,获取包含目标P的左右两幅图像,分别用L和R表示,由于双目摄像机必须确保前后位置相同,所以这两幅图像处于同一垂直空间平面内,所以$Z_{L}=Z_{K}$,P在左右图像上的投影坐标分别表示为$(X_{L},Y_{L})$和$(X_{R},Y_{R})$,又因为摄像机水平平整,所以有$Y_{L}=Y_{R}$。假设现实坐标系与摄像机左坐标系重合,那么P点与左摄像机中心光轴的距离为X,B表示两摄像机光轴中心之间的距离,所以P点与右摄像机中心光轴的距离为$(B-X)$

根据相似三角形的计算原理,可得公式2-1:
\begin{equation}
\frac{X_{L}}{X}=\frac{f}{Z-f}=\frac{X_{R}}{B-X}
\end{equation}
其中,目标距离Z和目标横坐标X都是未知数,变换公式可得公式2-2:
\begin{equation}
BX_{L}=(X_{L}+X_{R})X
\end{equation}
由公式2-2可解得观测目标横坐标X:
\begin{equation}
X=\frac{BX_{L}}{X_{L}+X_{R}}
\end{equation}
再将观测目标横坐标X值,即2-3代入公式2-1得:
\begin{equation}
\frac{X_{L}+X_{R}}{B}=\frac{f}{Z-f}
\end{equation}
令$X_{L}+X_{R}=D$,D即为视差,经过公式变化可得:
\begin{equation}
Z=\frac{Bf}{D}+f=f(1+\frac{B}{D})
\end{equation}
在公式2-5中,可以通过测量,标定获得所需要的系统参数,如像距f、轴间距B,视差$D=X_{L}+X_{R}$
通过立体匹配方法在两图像中找取对应点坐标计算现实场景中同一像素点的偏移量得到。完成标定和立体匹配后,f、B、D已知,就可以计算得目标点P与成像
平面之间的距离Z,即深度信息,将距离Z代入公式2-1,可得目标横坐标X:
\begin{equation}
X=\frac{B}{D}X_{L}
\end{equation}
同理可得,目标纵坐标Y为:
\begin{equation}
Y=\frac{B}{D}Y_{L}
\end{equation}
公式(2-5) (2-6) (2-7)分别求出了空间目标像素点在世界坐标下的坐标位置,实现双目系统的功能,这也就是双目立体视觉测距技术的基本原理。
如何准确的获取内部参数和视差图是重点,这也是确保最终测距精确度的难点,必须有足够准确的视差图和摄像机参数。下一节将介绍摄像机标定的过程,而视差图的获取将在下一章展开。

\section{摄像机的标定}

双目视觉系统在对获取图像信息进行加工得出深度信息的过程中,必须得到摄像机的一些内部固定参数,另外由于存在畸变,输入图像在进行立体匹配之前必须经过摄像机参数的校正,摄像机内外部参数获取的过程就是摄像机标定,摄像机标定结果直接影响立体匹配效果以及最终的信息还原情况。

摄像机标定主要包括摄像机成像系统内外的几何及光学参数的标定,双目摄像机之间相对位置关系的标定,其主要目的是确定目标物体的世界坐标与图像坐标的映射关系。摄像机标定的精度直接影响到双目视觉系统的精确度,是决定立体匹配效果的关键因素,具有重要的研究意义。

获取图像的每个像素点坐标与空间中物体表面对应点的实际位置具有相关性,这种位置上的映射关系由摄像机的成像几何模型唯一确定。通常把这个成像的几何模型参数称为摄像机参数,计算这些参数的过程也就是摄像机定标的过程。由上一节推导可知,这是双目视觉系统所依赖的公式计算的基础。

摄像机标定最初是在摄影测量学中逐渐研究发展起来的,摄影测量学提出了四个标定问题\citeup{A-1}:

(1)绝对方位:确定两个坐标系统之间的位置和方向等一系列传感器在绝对坐标系的坐标点的校正。

(2)相对方位:确定相对方向和位置在两个摄像头从预测的校准点在现场。

(3)外部方位:确定方向和位置,相机在绝对坐标系统中的投影校准点。

(4)内部方位:确定内部的几何参数,其中主要包括摄像机常数,位置的主要观察点,纠正透镜畸变。

这些标定问题最先起源于由通过航拍提取信息,创建相关地形地图的技术,是一个经典的摄影测量学难题。值得一提的是摄影测量学也包括双目视觉系统立体匹配和深度恢复的问题。

摄像机的标定存在许多方法,传统摄像机标定方法精度比较
高,但需要足够的标定模板,过程复杂繁琐;相比之下,基于主动视觉的摄像机标定方法
不仅不需要标定模板,而且算法简单且稳定性较高,不过设备比较昂贵,并且不能够用于摄像机运动状态不确定的情形;自标定法是一种利用环境的刚体性,依靠图像间的对应关系进行标定的方法,这种算法灵活性强,但是稳定性不高。 
对于大多数双目视觉系统而言,通常只需要在实验或实际应用之前将摄像机固定好,进行一次标定即可,计算出相关参数,实现离线标定。虽然只进行一次标定即可,但是由于标定结果对图像纠正和立体匹配效果有直接影响,所以标定过程要尽可能精确。

在一次离线标定的前提下,主动视觉标定法和自标定法就不如传统标定算法有优势了,传统的标定算法,如张正友标定法仅需要一个平面模板,对摄像机和模板的位置没有限
制。张正友标定法通过摄像机拍摄同一个平面靶标,多次改变靶标的位置和角度,提取相关信息,因为在改变靶标的过程中假定摄像机的内参数不变,因此可以根据多个平面数据计算出摄像机的内部参数。通常张正友标定法使用的是黑白棋盘格作为靶标,其作为靶标的优势是,棋盘格的特征,即黑白格形成的角点非常容易提取,而且特征点之间的位置关系是确定的。



