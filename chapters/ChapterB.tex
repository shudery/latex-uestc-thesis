% !Mode:: "TeX:UTF-8"

\chapter{双目视觉系统}

\section{双目视觉系统概述}
双目视觉系统如同人的双眼一样,我们观察外部世界,感知到外界三维信息,对物体远近的位置信息进行判断,在大脑中加工这些信息以使我们掌握真实的客观世界形态。双目视觉系统对人类双眼的模拟就是依据于此,为了使计算机能够获得客观世界中的物体三维信息,就需要利用双目视觉系统,将图像对的二维信息进行一系列处理,还原世界坐标下该物体的三维信息。

如果只有一幅图像,那么摄像机只能获取到图像中物体的二维信息,因为图像本来就是只能保存平面在某一时刻的信息,摄像机无法识别某一像素点在与摄像机平面的垂线方面上进行移动的过程,因此无法得到完整的物体信息,但是如果有两台摄像机,那么就可以捕捉到这些信息,双目摄像机由此而来。

双目立体视觉获取深度信息前需要求出图像的视差图,视差图就是两幅输入图像之间像素点的偏移量图,根据这些偏移量信息和摄像机标定得到的内部参数就能求出深度信息。视差图的获取的基本过程一共包括三个步骤:摄像机标定,获取成像模型所需内外部参数;图像获取,用标定好的摄像机系统拍摄目标场景图像,采集二维信息;立体匹配方法,寻找输入图像对之间的对应像素点,输出视差图;最后就是根据视差计来算深度,获取物体深度信息。综上可以按照功能可以将双目视觉系统分为四个主要的模块:图像获取,摄像机标定,立体匹配,深度计算,如图\ref{2-3}所示。

\pic[h]{双目视觉系统的主要模块}{width=14cm}{2-3}

\section{双目视觉系统的基本原理}

人类获取目标深度信息的原理如图\ref{2-1}所示。图中,L和R分别代表人的左眼和右眼以及两眼的视网膜图像,B为两眼间的瞳距(一般人眼的瞳距约为6厘米),
P为观察的目标。由于瞳距的关系,两只眼睛同时观察同一目标,由于两只眼睛的观察
角度不同,所以会有左眼看到左边多一些、右眼看到右边多一些的情况。若两眼同时观
察目标点P,则点P分别投影成像于两眼,这两个投影是有位置差异的,这个差异就是
视差,场景中所有点的视差组成一幅图像,称为视差图。大脑对视差图进行处理,可以产生对
场景的立体感知。人眼视觉的这个立体感知的过程就是基于视差的原理,视差原理也是计算机视觉的原理基础。

\pic[h]{人类视觉感知深度的原理示意图}{width=11cm}{2-1}

基于人类视觉特性的研究是立体视觉技术的重要研究方向,趋近于人类视觉能力是立体视觉技术的研究目标目前,在计算机立体视觉技术的研究中,基于单目多焦距图像序列的立体视觉技术、基于人类感知特性的立体视觉
技术以及基于双目成像的立体视觉技术都有研究人员从事相应工作。根据环境条件需求
和重构效果需求的不同,选择的立体视觉技术也不同。其中,双目立体视觉技术是最接
近人类视觉系统、适应于机器实现且重构效果较好的立体视觉技术。

双目视觉系统模拟人的视觉,对物体按照一定的规格采集两幅图像,以此分析计算出视差图。如图\ref{2-2}所示,对物体的某一点的深度感知无法在只有单目摄像机时感知,当物体从P点移动到Q时,对于做摄像机而言物体没有发生变化,而右摄像机则记录下了深度的信息,根据几何成像的原理即刻得出物体深度变化信息。而左右摄像机匹配对应像素点的过程就是立体匹配方法,将在下一章详细介绍。

\pic[h]{双目摄像机获取像素点信息的示意图}{keepaspectratio=false,height=6cm}{2-2}

\section{双目视觉的数学模型}

为了便于理解和计算方便,通常成像面翻转至视点与目标点之间,这样视点和目标

点连线与两成像面的交点即为目标点在成像面的像,如图\ref{2-4}所示为经翻转成像面的双
目测距原理示意图。
调节与标定完成后,两台摄像机同时对焦距范围内待测场景目标P拍摄,分别获得
L和R两幅图像,两幅图像在同一个与Z轴垂直的空间平面,即$Z_{L}=Z_{K}$,两幅图像中P
的对应点(即P在两成像平面的投影)坐标分别为$(X_{L},Y_{L})$和$(X_{R},Y_{R})$,首先以X
轴方向为例推导目标距离,令$Y_{L}=Y_{R}$。假定世界坐标系与左成像图坐标系重合,则P点
与左摄像机光轴距离为X,与摄像机光轴距离为$(B-X)$。

\pic[h]{双目测距原理图}{keepaspectratio=false,height=6cm}{2-4}

根据相似三角形计算原理,可得公式2.1
\begin{equation}
\frac{X_{L}}{X}=\frac{f}{Z-f}=\frac{X_{R}}{B-X}
\end{equation}
公式中,目标距离Z和目标横坐标X是未知数,公式变换可得公式2.2
\begin{equation}
BX_{L}=(X_{L}+X_{R})X
\end{equation}
由此可解得目标横坐标X,如公式2.3所示
\begin{equation}
X=\frac{BX_{L}}{X_{L}+X_{R}}
\end{equation}
再将目标横坐标X (即公式2.3)代入公式2.1得公式2.4
\begin{equation}
\frac{X_{L}+X_{R}}{B}=\frac{f}{Z-f}
\end{equation}
令$X_{L}+X_{R}=D$,经公式变化可得公式2.5
\begin{equation}
Z=\frac{Bf}{D}+f=f(1+\frac{B}{D})
\end{equation}
在公式2.5中,像距f、轴间距B是系统参数,可以通过测量获得,视差$D=X_{L}+X_{R}$
可以在两图像中找取对应点坐标计算得到。f、B、D已知,即可计算得目标点P与成像
平面之间的距离Z,将距离Z (即公式2.5)代入公式2.1,可得目标横坐标X如公式2.6所示
\begin{equation}
X=\frac{B}{D}X_{L}
\end{equation}
同理,目标纵坐标Y如公式2.7所示
\begin{equation}
Y=\frac{B}{D}Y_{L}
\end{equation}
公式(2.5) (2.6) (2.7)组成空间点P的坐标。这就是双目立体视觉测距的基本原
理。该方法实现点测距的前提是已准确完成对应点匹配,通过两对应点的坐标实现点测
距。当待测场景的大多数点完成测距,以其三维坐标构建空间坐标系,即可实现三维重
建


\section{摄像机的标定}

双目视觉系统根据摄像机获取的图像信息计算出待测点的深度信息,重建目标场景的表面三维形状与空间位置。三维空间点与其在两二维成像平面中的投影点是对应的,该对应关系取决于摄像机参数,这些参数需要通过实验和计算来确定,该过程就称为摄像机标定。实现三维重建首先需要采集准确的目标场景图像,摄像机标定正是获取准确图像的必要过程。

摄像机的标定最初在摄影测量学中研究和发展,摄影测量学提出了四个标定问题:

(1)绝对方位:确定两个坐标系统之间的转换或位置和方向的一系列传感器在绝对坐标系的坐标点的校正。

(2)相对方位:确定了相对位置和方向在两个摄像头从预测的校准点在现场。

(3)外部方位:确定位置和方向,相机在绝对坐标系统的投影在现场校准点。

(4)内部方位:确定内部几何相机,其中包括摄像机常数,位置的主要观点,并更正为透镜畸变。

这些标定问题源于由航拍照片创建地形地图的技术,是经典的摄影测量学问题。除了这四个基本标定问题,摄影测量学也涉及双目立体场景差异深度确定和立体影像重建。

摄像机标定包括摄像机成像系统内外几何及光学参数的标定和多个摄像机之间相对位置关系的标定,其目的是确定目标物体世界坐标与图像坐标之间的映射关系。摄像机标定的精度直接关系到计算机视觉系统的系统误差,是决定三维重建精度的关键因素,非常具有研究和改进意义。

图像上每点的坐标与空间中物体表面对应点的实际位置相关,这种位置上的映射关系是由摄像机的成像几何模型所唯一确定的。一般把这个几何模型的参数称作摄像机参数,计算这些参数的过程就是摄像机定标的过程。作为双目立体视觉技术赖以计算的基础,摄像机标定的精度直接影响整个系统的测量效果。



