% !Mode:: "TeX:UTF-8"

\chapter{绪论}
\section{计算机视觉技术概述}
\subsection{计算机视觉简介}
人类和动物通过自身的视觉系统,可以获得外界物体的形状大小、明暗亮度、行为变化、位置、颜色等各种对交际发展,机体生存具有重要意义的信息。据统计至少有80\%以上的外界信息经由视觉系统获得,由此可见视觉是人和动物最重要的感觉。随着科学技术的不断发展,视觉理论框架的提出和完善,计算机硬件和摄像机采集能力的提高,人们逐渐开始通过利用计算机及相关的其他设备模仿人类的视觉系统,利用它对采集的图片或视频进行处理,经过一系列的算法运算,实现机器对客观世界中的三维场景进行感知、识别和理解,这就是计算机视觉。

人类视觉的主要功能可概括为:适应外界环境,控制自身的运动。相应的,计算机视觉系统需要:能够准确识别物体,确定物体的位置和形状以及判断物体的运动状态。因此,目标的识别、定位、三维形状建模以及运动状态分析就组成了计算机视觉的主要研究内容。计算机视觉是科学领域中的一个极其富有挑战性的重要研究课题和研究领域。

当我们对现实中的某个物体进行肉眼的识别时,人类的大脑需要对这个物体存在某种表达。目前,计算机视觉界主要存在两种主流的表达理论。一种是上世纪八十年代初马尔(Marr)提出的三维表达理论(三维重建理论),另一种是本世纪初发展出的基于图像的表达理论。三维表达理论的基本思想是:物体是以其几何形状的方式在大脑中被表达的。由于物体的三维形状和观察角度无关,所以三维表达是一种与视角无关的表达,对此,Marr在其同时期出版的书中作了详细介绍,可以说正是 Marr 的三维物体表达模型打开了计算机视觉研究的大门。但随着研究的深入,人们发现 Marr  的三维重建理论与很多生理和心理实验有出入,针对这些问题,一种基于图像表达的全新
理论被提了出来。2004 年 Poggio  等人对基于图像表达的理论做出了比较全面的论述。时间再往后,随着神经科学研究的发展,基于图像表示方式的弊端也渐渐凸显,一些学者开始研究将二维和三维混合起来建模。综上,物体表达是计算机视觉中避免不了但至今仍没有得到很好解决的一个任务。 一般来说,计算机视觉的发展有马尔视觉计算理论、目的视觉理论、分层重建理论和基于学习的视觉理论这 4 个历程。而无论在哪个阶段,对三维世界的精确记录都是计算机视觉研究的基础。

\subsection{计算机视觉的发展现状}

计算机视觉学作为一门新兴学科,自二十世纪中期开始迅猛发展,成为当时最为热门的研究主题之一。计算机视觉使用计算机及相关的其他设备对生物视觉进行仿真和模拟,它的重要功能和主要任务是通过对采集的图片或视频进行处理以获得相应场景的三维信息,就像人类和许多其他类生物每天所做的那样。由于计算机视觉在工农业生产、地质学、虚拟现实、航天航空、天文学、气象学、医学及军事并学等领域都发挥了很大的作用以及扮演着非常重要的角色,因此越来越受到学者们的重视。

如今,经过众多学者的不懈努力,己经形成了一整套系统的理论与算法,并且得到了迅速的发展和广泛的研究。计算机视觉技术已经具备了通过计算机来对数字图像信息进行各种各样地处理和加工的功能。集数字图像处理、光学、几何学、物理学、模式识别及数字信号处理等知识于一体,计算机视觉技术作为一门迅速发展起来的科学领域,计算机视觉已经涉及到图像处理、计算机图形学、计算几何等诸多应用领域。

\subsection{计算机视觉的关键问题}

在客观现实空间中,物体深度信息的获取是计算机视觉技术研究领域中的核心解决部分。目前随着计算机视觉技术和立体匹配技术的迅速发展,为了能够更加准确,快速地获取真实空间中景物的深度信息,已经研究出很多方法和理论,并逐渐成为了立体视觉领域中的一个热门研究主题。在立体视觉技术中,双目立体视觉技术就是主要针对以上问题的一个专门领域,通过对真实场景不同角度的图像的分析和匹配,从而获取真实场景中物体的视差信息,最后通过物体的视差信息和深度信息的关系,计算出物体的深度信息,为进一步的三维重建、机器人导航等热门应用领域的应用做好准备。因此,无论是从视觉生理的角度还是在工程应用中,双目立体视觉的研究都具有十分重要的意义。 

双目视觉是计算机视觉的一个基础而重要的话题,而在双目视觉系统的实现模块中,立体匹配方法决定着计算机匹配精准度和匹配速度,是直接影响到双目视觉系统的匹配效果最重要的因素。可以说立体匹配方法研究是双目视觉领域中的最核心问题。按照提取的视差图的稀疏程度,立体匹配方法主要有基于灰度信息的匹配算法和基于特征信息的匹配算法。

\section{双目视觉系统简介}

双目立体视觉(Binocular Stereo Vision),简称双目视觉,是计算机视觉的一项重要研究内容。双目立体视觉技术模拟人眼视觉特性,从两个不同的角度观察同一个场景的同一目标,分别获得其投影图像,匹配对应点并获取偏差信息,根据三角测量原理,利用该偏差信息计算对应点的距离信息,从而获取景物的三维信息。

一个完整的双目立体视觉系统通常由相机安装、摄像头标定、立体校正、图像预处理、立体匹配和深度信息计算这几个部分组成。

\section{立体匹配技术简介}

立体匹配是由场景中同一物体的两幅或多幅图像获取物体三维几何信息的一种技术。它是这样的一个过程:以三维场景中同一物体不同角度拍摄得到的两幅图像作为输入,通过图像配准,寻找真实三维场景中某一物体在两幅图像中对应的像素点的过程。立体匹配通常包括匹配代价计算、匹配代价聚集、视差最优化计算和视差细化四个步骤。

按照提取的视差图的稀疏程度,立体匹配方法主要有基于灰度信息的匹配算法和基于特征信息的匹配算法。根据匹配基元的不同,主要分为基于特征,基于相位,基于区域匹配多个不同方向。基于特征的立体匹配方法受图像特征影响大,并且只能获得稀疏的视差图,匹配的准确度较差;基于相位的匹配方法算法复杂度较大,匹配时间过长,不适用于需要快速匹配的场景;根据优化规则还可分为全局匹配和局部匹配方法,局部匹配算法和基于区域匹配算法其实是相同的原理,只是分类的出发角度不同。

立体匹配算法必须解决以下三个问题:选择正确的匹配特征,寻找特征间的本质属性及建立能正确匹配所选特征的稳定算法。立体匹配在立体视 觉中占有重要的作用,立体视觉的应用推动匹配算法不断的发展, 尤其是在速度和精度两个方面,文章简要分析了匹配算法的分类,较详细地综述了匹配算法在实时性和准确性两个方面的进展。

\section{本课题主要研究内容}

本课题研究计算机视觉领域中的关键问题:双目视觉立体匹配方法。基于双目视觉系统成像原理,要实现双目视觉系统的基本模块,包括摄像机的离线标定,图像获取,立体匹配,核心是研究立体匹配方法的实现。

局部匹配算法是以两幅图像对应像素点的邻域为参考单元进行相似度匹配的方法,局部匹配算法既能得到稠密的视差图,又能极大地缩小搜索范围和运算时间。考虑到实验条件和算法在现实场景中的实用性,本课题研究的立体匹配算法主要是基于区域的立体匹配方法,即局部匹配算法。

本课题将在实现经典的局部固定窗口算法基础上,从匹配窗口大小,遮挡图像检测,自适应权值等多个方向优化算法以提高立体匹配的效率和精准度,利用匹配评估函数对比算法的匹配结果,得到匹配效果良好的立体匹配方法,最后通过摄像机采样实际图片作为输入,检验匹配效果。

\section{本文结构安排}

文章结构将由整体到局部,依次对双目视觉系统的框架原理,系统各个模块的实现方法,立体匹配技术实现,不同场景和图像情况下的算法优化方法,评估匹配函数,实际场景匹配效果试验进行分析和展现。具体章节结构安排如下:

第 1 章:绪论,简单介绍计算机视觉的发展现状,以及其核心问题双目视觉立体匹配技术的一些概念。 

第 2 章:双目视觉系统的基本框架和实现方法,立体视觉标定,讨论离线的摄像机标定技术,以及畸变模型及参数的计算,设计实验验证相应算法的可行性。 

第 3 章:介绍多种立体匹配方法,分析对比其各自的实现难点,着重解决的图像匹配问题,匹配效果和实际应用的可行性。实现立体匹配基础算法,针对不同的图像情况,讨论多种可执行的优化方案,评估对比不同方法的匹配效果,提高立体匹配算法的匹配精确度和执行速度。

第 4 章:构建双目视觉系统,采用实际场景的图像对前面研究的算法匹配效果进行验证。最后,总结和展望,对全文的工作进行总结,提出未来进一步改进和研究的方向。 



