% !Mode:: "TeX:UTF-8"

\chapter{绪论}
\section{计算机视觉技术概述}
\subsection{计算机视觉简介}
众所周知,生物可以通过自身的视觉系统,取外界景象的形状大小、明暗亮度、行为变化、位置颜色等各种信息,这一功能对生物的自身发展,行为,心理行为甚至生存具有重要意义。据统计人类获取的信息至少有超过五分之四的外界信息是经由视觉系统获得,这也就是为什么说视觉系统对于信息采集至关重要。随着科学技术的不断发展,为了让机器能够获得更加丰富的外界信息,从而为人类提供更加可靠高效的服务,提出了视觉理论的框架,并不断地完善计算机视觉的相关知识架构,加上计算机硬件和摄像机采集能力的提高,人们逐渐开始通过计算机及相关的其他设备模仿人类的视觉系统。实现具有图像信息获取能力的系统,利用它对采集的图片或视频进行处理,经过一系列的算法运算,实现机器对现实中的各种应用场景进行分析判断,提取有用信息并根据这些信息进行对应的机器决策,这就是计算机视觉。

人类的视觉系统不外乎这两个主要的功能,第一是适应外界环境,第二是根据信息决定自身行为。相应的,要实现一个计算机视觉系统,该系统必须能够准确识别图像信息,确定物体的位置和形状以及根据一段时间内物体的位置判断物体的运动状态。因此,目标的识别定位、三维形状建模以及运动状态分析就组成了计算机视觉的主要研究内容。计算机视觉是科学领域中的一个重要研究领域,它所具备的科研价值与其巨大的应用潜力是相匹配的。

由于是对人类视觉的模仿,那么要实现计算机视觉,就要分析提炼出人类视觉系统的识别过程,每当我们用眼睛接收到现实世界中物体的信息时,人类的大脑需要对这个物体存在某种表达。目前,计算机视觉领域中主要存在两种主流的表达理论。一种是由视觉理论研究的创始人马尔(Marr)提出的三维表达理论(三维重建理论)\citeup{1-1},马尔从信息处理的角度出发,综合了心里物理学、神经生理学和图像处理
学的研究成果,首次提出了众所周知的 Marr 视觉理论。它也是最早的计算机视觉理论框架。Marr 的视觉理论其实就是把计算机视觉的过程看作一个信息处理过程,并将这个过程分为三个阶段层次:计算理论层次、算法与数据结构层次和硬件实现层次。另一种是本世纪初发展出的基于图像的表达理论。三维表达理论的基本思想是:物体是以通过几何形状的方式在大脑中被表达的。由于改变物体的观察视角是不会改变物体的三维形态的,所以三维表达是一种与视角无关的表达,对此,Marr在其同时期出版的书中作了详细介绍,可以说正是 Marr 的三维物体表达模型打开了计算机视觉研究的大门。但随着研究的深入,人们发现 Marr  的三维重建理论与很多生理和心理实验有出入,针对这些问题,一种基于图像表达的全新
理论被提了出来。2004 年 Poggio  等人对基于图像表达的理论做出了比较全面的论述。时间再往后,随着神经科学研究的发展,基于图像表示方式的弊端也渐渐凸显,一些学者开始研究将二维和三维混合起来建模。综上,物体表达是计算机视觉中避免不了但至今仍没有得到很好解决的一个任务。 一般来说,计算机视觉的发展有马尔视觉计算理论、目的视觉理论、分层重建理论和基于学习的视觉理论这 4 个历程。而无论在哪个阶段,对显示场景的精确记录都是计算机视觉研究的基础。

\subsection{计算机视觉的发展现状}

计算机视觉是近代的一门新兴学科,伴随计算机技术的快速发展,在上世纪中后期开始迅速发展,逐渐成为近代计算机科学最为热门的研究主题之一。计算机视觉指的是利用计算机和相关摄像设备来仿真和模拟生物的视觉,通过对采集的图像或视频信息进行处理,计算出相应场景的三维信息,这是计算机视觉的核心任务。如同生物视觉系统的工作机制,让机器也拥有视觉信息获取的能力,通过这些信息帮助机器进行更加准确的判断,让机器代替人类从事一些基础的服务工作。计算机视觉在许多领域有着广泛的应用场景,例如在地质勘测、交通监测、犯罪监控、增强现实、虚拟现实、人工智能等领域都是项目的核心技术,而且占据研究力度的比例越来越高,因此各领域的研究人员对计算机视觉技术的重视程度也是日渐提高。

科学技术不断发展,到了二十一世纪,在各领域学者的共同努力之下,计算机视觉逐渐形成并完善了一套系统的理论框架,如何快速实现一个具备视觉识别功能的系统已经逐渐体系化。计算机视觉技术已经可以通过计算机和相关的摄像机设备来对各种各样的视频图像信息进行识别,提取,匹配,处理了。围绕着计算机视觉系统实现的四个模块:图像提取,摄像标定,立体匹配,信息重建有了较好的实现方案,其中立体匹配作为双目视觉中的核心问题,是目前业界共同努力的方向,主要围绕如何提高立体匹配方法的匹配精确度和算法的运行速度这两个方面。

\subsection{计算机视觉的关键问题}

在客观现实空间中,物体深度信息的获取是计算机视觉技术研究领域中的核心解决部分。目前随着计算机视觉技术和立体匹配技术的迅速发展,为了能够更加准确,快速地获取真实空间中景物的深度信息,已经研究出很多方法和理论,并逐渐成为了立体视觉领域中的一个热门研究主题。双目视觉通过立体匹配方法对图像信息进行一些列处理,还原出显示世界坐标系中的深度坐标,是计算机视觉中的核心问题,要想获得深度信息从而进行一系列后续信息处理和操作判断就必须做好双目视觉中最核心的一个功能,立体匹配。 

双目视觉是计算机视觉的一个基础而重要的话题,而在双目视觉系统的实现模块中,立体匹配方法决定着计算机匹配精准度和匹配速度,是直接影响到双目视觉系统的匹配效果最重要的因素。可以说立体匹配方法研究是双目视觉领域中的最核心问题。立体匹配方法根据分类基准的不同有多种方式,主要有注重匹配精确度而牺牲算法运算速率的全局匹配方法,反之也有注重匹配速度,对匹配精确度要求不高的局部匹配方法,它们有各自不同的应用场景。无论是哪一种匹配算法,都要解决图像匹配过程中的遮挡区域,低纹理区域,深度不连续区域的误匹配问题。

\section{双目视觉系统简介}

双目立体视觉(Binocular Stereo Vision),也简单称作双目视觉,是计算机视觉的基础研究分支。由于要获取图像的三维信息,必须至少用有两幅图像,单目系统在稳定性上也远远不如双目系统,所以相对于单目系统,双目系统更具有研究意义。类似人类拥有一双眼睛,双目系统对于同一物体可以获取不同角度的两幅图像,提取其像素信息,利用摄像机内部参数,根据匹配算法计算两幅图像之间像素点的对应关系,得出图像视差图,再根据三角测量原理,利用视差图信息计算出图像各个位置的三维深度坐标,这就是双目视觉系统的实现过程。

一个完整的双目立体视觉系统通常由相机安装、摄像头标定、立体校正、图像预处理、立体匹配和深度信息计算这几个部分组成。其中立体匹配是双目视觉系统实现的关键部分。

\section{立体匹配技术简介}

立体匹配是通过获取同一物体的两幅或者多幅图像的信息,计算物体三维几何信息的一种技术。立体匹配的基本过程是:首先获取现实场景中同一物体同一时刻不同角度的两幅图像,经过图像校准后,建立图像对同一世界坐标像素点之间的映射关系,最终得到匹配视差图。立体匹配主要包括四个步骤,匹配代价计算、匹配代价聚集、视差最优化计算和视差细化。

如果根据双目系统提取计算出的视差图稀疏程度分类,立体匹配方法主要有基于灰度信息和基于特征信息这两种匹配方法。还可以根据匹配基元的不同,将立体匹配分为基于特征,基于相位,基于区域匹配等多种各具特色的方法。基于特征的立体匹配方法特点是受图像特征影响大,并且该方法只能获得稀疏的视差图,匹配的准确度相对而言较差;基于相位的匹配方法算法复杂度则比较大,匹配时间往往过长,这就决定了它不适用于需要快速匹配的场景;另外根据优化规则的区别,还可分为全局匹配和局部匹配方法,局部匹配算法和基于区域匹配算法其实是相同的原理,只是分类的出发角度不同而已。

立体匹配算法实现的指导思想主要是需要实现以下三个目标:输出图像对视差图,以此为基础如何提高匹配精确度和算法速度。立体匹配在立体视 觉中占有重要的作用,立体视觉的应用推动匹配算法不断的发展, 尤其是在速度和精度两个方面,这也是立体匹配过程中要解决的一个时间和精度的矛盾。

\section{本课题主要研究内容}

本课题将研究计算机在视觉领域中的关键问题:双目视觉系统立体匹配方法。基于双目视觉系统的成像原理和立体匹配过程的学习,研究匹配算法,实现双目视觉系统的基本模块,包括摄像机的离线标定,图像获取,立体匹配,内容重点是研究立体匹配方法的实现。

上文已经提到,立体匹配方法是以两幅图像对应像素点的邻域作为参考单元,通过某一相似性函数进行相似度匹配的方法。在各种不同的立体匹配方法中,局部匹配算法既能得到稠密的视差图,又能极大程度地缩小搜索范围还有运算时间。考虑到实验条件和算法在现实场景中的实用性,本课题研究的立体匹配算法主要是基于区域的立体匹配方法,即局部匹配算法。

本课题将在实现经典的局部固定窗口算法基础上,从匹配窗口大小,遮挡图像检测,自适应权值等多个方向优化算法以提高立体匹配的效率和精准度,利用匹配评估函数对比算法的匹配结果,得到匹配效果良好的立体匹配方法,对比不同实例图像的匹配效果,得到稳定的匹配优化算法。

\section{本文结构安排}

文章结构将由整体到局部,依次对双目视觉系统的框架原理,系统各个模块的实现方法,立体匹配技术实现,不同场景和图像情况下的算法优化方法,评估匹配函数,实际场景匹配效果试验进行分析和展现。具体章节结构安排如下:

第 1 章:绪论,这一章主要是简单介绍计算机视觉的发展现状,以及其核心问题双目视觉立体匹配技术的一些概念。 

第 2 章:介绍双目视觉系统的基本框架和实现方法,立体视觉标定,研究双目视觉系统的数学模型,研究双目视觉系统中摄像机标定的过程及意义

第 3 章:介绍多种立体匹配方法,分析对比算法各自的实现难点,对不同图片区域的匹配精确度,以及实际应用的可行性。在介绍完立体匹配基础算法后,针对不同的图像情况,研究立体匹配过程中精确度和运算时间要达到一定的标准,所要解决的一些难点,为算法优化提供可行方向。

第 4 章:构建双目视觉系统,实现图像采集,离线标定,通过实验完成立体匹配算法,并以此为基础进行算法优化,对匹配结果进行测试,最终得到一种匹配效果较好的立体匹配算法。最后,总结实验结果和本文,对全文的工作进行总结,提出未来进一步改进和研究的方向。 



