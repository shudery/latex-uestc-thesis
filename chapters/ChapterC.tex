% !Mode:: "TeX:UTF-8"

\chapter{立体匹配方法及优化}

\section{立体匹配方法简介}
立体匹配是双目视觉系统的核心问题,通过立体匹配寻找两幅图像中对应的像素点的位置,得出匹配结果,通常用视差图表示,从而进一步求出深度信息。

从根本上讲,立体匹配算法主要是通过建立一个能量代价函数,通过此能量代价函数最小化来估计像素点视差值。立体匹配算法的实质就是一个最优化求解问题,通过建立合理的能量函数,增加一些约束,采用最优化理论的方法进行方程求解,这也是所有的病态问题求解方法。现实生活中,对物体进行拍摄时难免受到某些不确定环境因素干扰的影响,在不同角度下拍摄的同一场景的图片会有所不同。进行立体匹配时难以获得具有很高准确度的结果。为了最大限度地减小这些不利因素的影响,选择恰当的匹配算法是立体匹配技术的重点。

经过多年的研究和发展,已经出现许多各具优势的匹配算法,但是这些算法都不外乎又几个常见的部分构成。立体匹配方法通常可以分为匹配代价计算,匹配代价聚集,视差最优化计算,视差细化四个部分。
\pic[h]{立体匹配常见算法分类}{keepaspectratio=false,height=6cm}{3-1}

匹配代价计算:这一步最常用的方法有基于灰度值差值的绝对值的匹配代价计算以及基于灰度值差值平方的绝对值的匹配代价计算,还有基于可变权值的灰度值差值的绝对值的匹配代价计算。 

匹配代价聚集:通过累加邻域内的匹配代价或取其平均值来代替单个参考像素的匹配,就是匹配代价聚集,通常用这种方式来增加匹配结果信噪比。局部匹配方法主要通过代价聚集步骤来提高最终的匹配质量。近来,越来越多的全局匹配方法也利用代价聚集步骤来提高匹配质量,并且已经达到了不错的效果。聚集区域通常是具有固定视差的二维区域。窗口可以使用固定窗口、可变窗口、自适应形状窗口等。 

视差最优化计算:这一步根据上一步像素的匹配代价聚集结果来为每一个像素确定其视差值。局部匹配方法在视差计算步骤使用的方法很直接,只是简单的为每一个像素选择使得像素匹配代价最小的视差作为像素的最终视差,即赢者通吃算法(Winner Take All)。但是,这个方法有一个缺陷,那就是目标图像中的像素点可能被多次选作匹配点,因此可能造成很多的误匹配。

视差细化:这一步的目的是为了进一步降低生成的视差图的误匹配率,从而整体提高匹配视差图的质量。常用的一种方法是使用中值滤波,消除图像中的毛刺点,或者使用中间像素视差精炼法,通过曲线来拟合视差变化,从而消除可能的毛刺点。此方法适用于图像中像素灰度平滑变化的区域。

\section{立体匹配方法的分类和基本过程}

立体匹配方法经过多年的研究发展,涌现了各种各样具有不同优势的算法。

若根据匹配基元的不同,主要分为基于特征的立体匹配方法,基于相位的立体匹配方法,基于区域的立体匹配方法。基于特征的立体匹配方法中,匹配基元是基于全局优化的图像特征集,如点,边缘,区域特征,从图像特征集中寻找对应关系实现匹配;基于区域的立体匹配方法中,匹配基元是具有一定尺寸的模板窗口,釆用相关函数测量视差范围内待测窗口相似性实现匹配;基于相位的立体匹配方法,匹配基元是图像相位,利用多尺度的空间频率分析方法,提取图像不同频段的信息进行匹配。匹配原理基本相同,都是基于基元的相似性测量,但适用方法的搜索策略各不相同。

若根据匹配结果的稠密程度,可分为稠密匹配和稀疏匹配。稠密匹配以图像的灰度、邻域相关程度等为判断依据进行匹配,为绝大多数像素找到匹配结果。稀疏匹配是以边缘、轮廓、线段等图像特征为匹配内容进行匹配,也就是基于特征的立体匹配,仅得到特征点处匹配结果。

若根据匹配的优化规则,可以分为全局立体匹配方法和局部立体匹配方法。局部匹配方法在经典的窗口匹配基础上,还有自适应窗体,自适应权值,多窗口匹配等优化方向,局部匹配方法其实也是一种基于区域的立体匹配方法,它们只是分类角度不同而已。全局立体匹配的主要过程是在局部匹配的基础上进行匹配结果优化,主要的全局优化方法有图割法,置信传播,动态规划等。

\pic[h]{立体匹配常见算法分类}{keepaspectratio=false,height=6cm}{3-2}

本文将介绍三种比较常见的立体匹配方法:基于特征的立体匹配方法,全局立体匹配方法,局部立体匹配方法(基于区域的立体匹配方法)。

\subsection{基于特征的立体匹配方法}
像素灰度值是图像的基本特征,不过单一像素的灰度值不能提供足够的信息,无法实现立体匹配,而图像特征可以提供一个更高层次的图像描述。图像特征是一幅图像中局部的、有意义的、可检测的部分。特征的匹配算法,主要是基于几何特征信息(边缘、线、轮廓、兴趣点、角点和几何基元等),针对几何特征点进行视差估计,所以先要提取图像的特征点,尽而利用这些特征点的视差值信息来重建三维空间场景。

匹配所需要的主要步骤:图像预处理、提取特征、特征点的匹配获取稀疏视差图,如果想得到稠密的视差图,需要采用插值的方法。可获得稀疏的视差图,经差值估计可获得稠密视差图。可提取点、线、面等局部特征,也可提取多边形和图像结构等全局特征。缺点是特征提取易受遮挡、光线、重复纹理等影响较大;差值估计计算量大,对特征不明显的图像匹配准确度差,而且特征提取的过程也会影响了匹配的速度。

\subsection{全局立体匹配方法}
全局立体匹配算法主要是在局部立体匹配的基础上,根据约束条件,采用全局的优化理论方法估计视差,建立全局能量函数,通过最小化全局能量函数得到最优视差值。全局立体匹配算法将待匹配点的匹配代价函数和相邻像素之间由不同视差造成的相互影响关系函数组成。这种平滑函数反映在立体匹配上就是上的立体匹配相关约束条件。

全局匹配算法得到的结果比较准确,但是其运行时间比较长,不适合实时运行,近年来随着计算机计算能力的大幅上升有所改善。主要的算法有图割(graphcuts)、信念传播(belief propagation)、动态规划等算法。

\subsection{局部立体匹配方法}
局部立体匹配算法是立体匹配算法中发展迅速并且应用领域非常广泛的一类算法,局部立体配算法的实质是:通过直接对输入的图像对之间的彩色信息或灰度信息进行匹配,从而得到其中一幅图像对应的视差图。不同于基于特征的匹配算法得到的稀疏视差图,大多数局部匹配算法都能得到图像对应的稠密视差图,保留了图像中丰富的视差信息和深度信息。基于以上原因,本课题的研究方向定位局部立体匹配算法的研究和改进,而局部立体匹配算法中,匹配窗口尺寸和形状的选择是一个很重要的因素,一个大小形状合适的窗口对算法的结果影响很大。窗口不能太小,否则,窗口中包含的可用信息太少,导致无匹配的可能性增大。窗口也不能太大,否则,一方面严重
影响到匹配效率,一方面可能受到投影畸变的影响。

由于窗口的选择对局部立体匹配算法的效果影响很大,由此形成了以SAD算法为代表的基于窗口的局部立体匹配算法。除基于窗口的局部匹配算法以外,近年来也出现了一些其他的以基于Census变换的局部算法为代表的基于图像变换的局部匹配算法,还包括基于傅立叶变换的局部算法、基于Rank变换的局部算法等。此外还有基于自适应权值的局部立体匹配算法,这一类算法表面采用固定的匹配窗口进行匹配,实则是通过自适应的权值为每一个参考像素选择最佳形状和尺寸的匹配窗口,以ASW算法为代表,正是基于自适应权值的局部立体匹配算法将局部立体匹配算法推向了顶峰。

局部立体匹配算法可大致分为三类:基于窗口的局部立体匹配算法、基于变换的局部立体匹配算法和基于自适应权值的局部立体匹配算法。 

基于窗口的局部立体匹配算法:顾名思义,即匹配窗口的大小是固定不变的。是局部算法中最为经典也是最为简单的一类。算法简单易于实现,而且匹配效率很高,缺点是算法提取的视差图受窗口大小影响很大,而且视差图中物体的轮廓不清晰。

基于变换的局部立体匹配算法:在进行匹配之前,先对匹配窗口进行变换,对经过变换的窗口进行立体匹配,代表算法有基于Census变换的局部立体匹配算法。其他的算法还包括:基于傅立叶变换的局部算法、基于Rank变换的局部算法等。由于该类算法在基于窗口的局部算法的基础上加了图像变换,因此算法复杂度和基于窗口的局部算法略高。由于以上原因,这一类算法效率较高,但提取的视差图中噪点仍然较多,物体边缘轮廓不够清晰。 

基于自适应权值的算法为每一个像素寻找匹配像素点时,不是单独地进行对应窗口像素的灰度值的差值作累积,而是为窗口中的每一个像素对中心像素的贡献设置一个权值(常用的权值有基于颜色的权值和基于距离的权值),这样,可以大大提升图像的整体匹配效果,尤其是在图像的边缘轮廓区域,获取较为清晰的视差图。在匹配效率方面,由于采用固定大小的窗口,因此匹配效率也相对于可变窗口的匹配算法效率高,因此,在双目立体匹配领域被广泛的使用。

局部立体匹配算法使用简单的赢者全胜(Winner Take  All)的视差选择策略,与全局立体匹配算法相比,它通过匹配窗口的全部像
素的累积量来进行匹配。相比全局立体匹配算法,局部立体匹配算法有以下优点: 

局部算法复杂度低:目前越来越多的应用环境对系统的响应时间要求很高,甚至很多应用场景要求算法的效率要接近实时运算,因此,虽然全局算法的匹配效果不错,但单从效率上讲,已经是毫无用武之地。鉴于这个原因,局部算法的研究和学习也越来越广泛和深入。 

局部算法匹配效果越来越接近全局算法:早期,全局算法备受关注,原因是全局算法获取的视差图质量相比当时的局部算法获取的视差图表现出色很多,但是随着局部算法的进一步研究和改进,目前,许多局部算法提取的视差图的效果已经很好,甚至超出某些全局算法的匹配效果,随着局部算法匹配效率的提高,相应的匹配复杂度也不断提升,即局部算法效果的提升是以牺牲匹配效率为代价的。尽管局部算法的复杂度不断提升,但是相比于全局算法,在速度方面的优势仍旧是非常显著的,因此,在某些对实时性和匹配效果并求的应用场景下,局部算法无疑是最佳选择。

\section{立体匹配的难点和算法优化方向}
在双目系统的立体匹配过程中,通常对于左图像中的某一个像素点,通过匹配方法找到右图的匹配候选点往往有多个,而实际上真正的匹配点最多只有一个,遮挡区域的像素则实际上是不能找到匹配点的。而产生多个匹配候选点则会提高误匹配的概率,导致匹配准确度的下降,在图像的遮挡区域,低纹理区域以及深度不连续区域是最容易产生误匹配的地方,这也是立体匹配需要克服的几个难点。也是本文算法优化所要解决的问题。

\pic[h]{局部匹配的难点}{keepaspectratio=false,height=6cm}{3-5}

遮挡区域:由于遮挡等因素,某些区域的像素点不可能同时出现在输入的两幅图像中,也就不可能找到对应的匹配点。遮挡是立体匹配中面临的困难之一,由于前景和背景在不同观察点观察的时候,位置偏移量会有所不同,部分的背景在参考图像上存在,而在目标图像中被遮挡了,使得参考图像中的像素点失去了对应的匹配点,从而造成误匹配。遮挡像素的检测,可以通过多幅图像进行处理,也可以利用交叉检验的方法判断被遮挡像素点,主要是采用后者进行检测遮挡的像素点。 

低纹理区域:该区域内的像素高度相似,通过一般的相似性测量来寻找匹配点,通常会找到多个匹配点与之对应,导致难以得到正确匹配结果。弱紋理或重复纹理区域主要是指在某些区域中的像素点,其颜色特征比较相近,即纹理特征并不丰富。在匹配的过程如果窗口的过大会导致在低纹理区域匹配过程中产生多个匹配候选点,可以通过对匹配窗口的筛选来改善匹配效果。

深度不连续区域:这种区域普遍存在于场景中的物体的边缘处,匹配时很容易产生边界不清晰问题。一般认为连续性约束只适用于物体表面,在物体边缘处是不满足的。针对这类区域中的像素,要提高匹配精度,可以通过采用多个窗体,在每一个窗体中,像素不是总处于窗体的中间位置,而是在不同的位置。然后计算在每类窗体下,计算匹配代价,为每一个像素选择一个最优的窗口。 

下一章将通过具体的实验环境,实现双目视觉系统的匹配功能并将优化局部立体匹配算法的匹配结果进行分析比较。
